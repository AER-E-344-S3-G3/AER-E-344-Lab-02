\chapter{Discussion} 
\label{cp:discussion}
Using, equations <equations for $q_T$ and $Delta P$>, we generated Figure <$q_T$ vs $Delta P$ graph>. From inspection, the data is very linear, suggesting a direct relationship between the dynamic pressure in the test chamber, $q_T$, and the change in pressure from points $A$ and $E$, $\Delta P$. After calculating a linear regression on this data using the `polyfit` function in MATLAB, we confirmed that there is a strong direct relationship since the coefficient of determination, $R^2$, is 0.9993. From the linear regression, we were also able to determine the $K$ coefficient, defined in Equation <$K$ definition equation>. This coefficient allows us to quantify the dynamic pressure—and furthermore, the flow speed—in the test chamber given only the static pressure readings from points $A$ and $E$. \par
Using the dynamic pressures we calculated and the definition of dynamic pressure, Equation $q=1/2 \rho V^2$, we calculated the velocity in the test chamber. Plotting the test chamber velocity, $v_T$, with respect to the frequency of the motor, $\omega _{motor}$, resulted in Figure <$v_T$ vs $\omega _{motor}$ graph>. Once more, we calculated the linear regression. The coefficient of determination was 0.9992 and the subsequent line of best fit is: 
\begin{center}
    $v_T = 1.32 * \omega _{motor} - 0.431 $
\end{center}

This equation allows us to directly determine the flow speed in the test chamber given only the frequency of the motor. Similarly, reversing this equation, 
$\omega_{motor} = 0.755 * v_T + 0.326$ allows us to determine the appropriate motor frequency for a given velocity. \par
This calibration is necessary to accurately predict the air speed in the test chamber, a critical value in wind tunnel testing. In many instances, inserting a pitot tube near the model or putting holes in the model could result in different flow characteristics (\url{https://www.aere.iastate.edu/~huhui/teaching/2024-01S/AerE344/class-notes/AerE344-Lecture-04-Pressure-Instrument.pdf}). Even if a pitot tube or pressure transducer were inserted further upstream in the wind tunnel—far enough that any turbulent effects due to the obstruction had dissipated by the time the flow reached the test chamber—the energy and dynamic pressure loss due to the boundary layer effect near the walls of the wind tunnel would result in inaccurate results. \par
By determining the relationship between the dynamic pressure in the test chamber and the pressure differential between points $A$ and $E$, only points $A$ and $E$ need their static pressure measured using the static pressure ports, minimizing adverse effects on the flow without sacrificing accuracy. \par
In future experiments, to increase the accuracy and precision of the pressure readings, an electronic manometer could be connected to the static pressure points at $A$ and $E$. Assuming point $A$ was connected to the main port and $E$ to the secondary port, the resulting voltage would be the pressure differential between point $A$ and $E$. Using the value of $K$ calculated in this lab, the velocity in the test chamber for a given pressure differential could be estimated trivially by multiplying the pressure differential by $K$. 