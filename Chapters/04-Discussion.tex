\chapter{Discussion} 
\label{cp:discussion}
From inspection of \autoref{fig:dynamic_pressure_graph}, the relationship between \gls{q_T} and \gls{DeltaP} is very linear, suggesting a direct relationship between the dynamic pressure in the test chamber, \gls{q_T}, and the change in pressure from points A and E, \gls{DeltaP}. After calculating a linear regression on this data using the \verb|polyfit()| function in \acrshort{matlab} (see \autoref{sec:determine_calibration_constant}), we confirmed that there is a strong direct relationship since the coefficient of determination, \gls{R^2}, is \num{0.9993}. From the linear regression, we were also able to determine the \gls{K}-coefficient, defined in \autoref{eq:K_def}. This coefficient allows us to calculate the dynamic pressure—and furthermore, the flow speed—in the test chamber given only the static pressure readings from points A and E.

Plotting the test chamber velocity, \gls{v_T}, with respect to the frequency of the motor, \gls{omega_motor}, resulted in \autoref{fig:velocity_graph}. Once more, we calculated the linear regression (see \autoref{sec:calc_velocity}). The coefficient of determination was \num{0.9992} and the subsequent line of best fit is shown in \autoref{eq:velocity_regression}.

\autoref{eq:velocity_regression} allows us to directly determine the flow speed in the test chamber given only the frequency of the motor. Similarly, reversing this equation, 
$\omega_{motor} = 0.755 * v_T + 0.326$, allows us to determine the appropriate motor frequency for a given velocity. For example, if the tester wanted to set the wind tunnel to \qty{15}{\meter\per\second}, they would set the wind tunnel motor frequency to $\omega_{motor} = 0.755 * 15 + 0.326 = \qty{11.7}{\hertz}$.

This calibration process is necessary to accurately estimate the airspeed in the test chamber, a critical value in wind tunnel testing. In many instances, inserting a pitot tube near the model or putting holes in the model could result in different flow characteristics \citep{lecture4-notes}. Even if a pitot tube or pressure transducer were inserted further upstream in the wind tunnel—far enough that any turbulent effects due to the obstruction had dissipated by the time the flow reached the test chamber—the energy and dynamic pressure loss due to the boundary layer effect near the walls of the wind tunnel could lead to inaccurate estimates. By determining the calibration constant, \gls{K}, pressure transducers need only be inserted normal to the flow at points A and E, minimizing adverse effects on the flow without sacrificing accuracy.

In future experiments, to increase the accuracy and precision of the pressure readings, an electronic manometer could be connected to the static pressure ports at point A and E. Assuming the port at point A was connected to the main port of the manometer and the port at point E was connected to the secondary port of the manometer, the resulting voltage would be the pressure differential between points A and E. Using the value of \gls{K} calculated in this lab, the dynamic pressure in the test chamber for a given pressure differential could be estimated trivially by multiplying the pressure differential by \gls{K}. If desired, the corresponding airspeed could be calculated using \autoref{eq:velocity}.
