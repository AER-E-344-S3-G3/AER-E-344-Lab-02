\chapter{Introduction}
\label{cp:introduction}
In our lab this week we learned about how dynamic pressure is essential for conducting experiments in wind tunnels and how it is not always practical for engineers to insert pitot-static tubes into the flow of the wind tunnel. One solution to this is to make pressure measurements at two points in the wind tunnel and then we can relate them using the dynamic pressure in the test section. The two points that we will be using to judge the pressure in the wind tunnel are located at A and E points in figure below.  

{\it insert figure}

The pressure difference in theory is a linear relationship to the dynamic pressure in the test section. So in this lab our goal is to find that calibration relationship so then we can use that feature to the wind tunnel. (Lab 2 manual: \url{https://www.aere.iastate.edu/~huhui/teaching/2024-01S/AerE344/lab-instruction/AerE344L-Lab-02-instruction.pdf}) 