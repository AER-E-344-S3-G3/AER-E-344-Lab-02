\thispagestyle{plain} % Page style without header and footer
% \pdfbookmark[1]{Resumo}{resumo} % Add entry to PDF
% \chapter*{Resumo} % Chapter* to appear without numeration
% \blindtext

% \keywordspt{Keyword A, Keyword B, Keyword C.}

% \blankpage

\pdfbookmark[1]{Abstract}{abstract} % Add entry to PDF
\chapter*{Abstract} % Chapter* to appear without numeration

The low-speed wind tunnel at Iowa State University is operated via a remote which controls the frequency of a motor connected to the wind tunnel. Since the motor frequency is not indicative of flow speed, to perform meaningful aerodynamic analyses, we first determined the calibration constant, \gls{K}—a coefficient that relates the dynamic pressure in the test chamber to the static pressure measured from two ports upstream of the test section. The calibration constant—which we determined to be $K=1.08$—enables us to convert the static pressure readings from upstream of the test section into the dynamic pressure of the test section and, subsequently, the air velocity in the test section. Our results indicated a linear relationship for both the dynamic pressure vs. static pressure differential relationship and the air speed vs. motor frequency relationship. The \gls{K}-coefficient determined in this lab will be used to calculate the dynamic pressure and airspeed in future labs.

% \blankpage


