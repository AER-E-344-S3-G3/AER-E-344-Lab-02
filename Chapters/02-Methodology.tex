\chapter{Methodology}
\label{cp:methodology}
\noindent\textbf{Apparatus:} \par
For this experiment, we had access to a low-speed wind tunnel. The tests conducted were within the frequency range of 0-40 Hz of the motor. A pitot tube was installed in its testing section, positioned at the center of it. \par
Four tubes acting as manometers with water as the liquid are connected to 2 different sections of a wind tunnel and to the pitot tube. Tubes 1 and 2 are connected to point A and E respectively (as seen in Figure <figure from introduction>). Tube 3 measures the total (or stagnation) pressure and tube 4 measures the static pressure from the pitot tube. A fifth tube with the same amount of water is used as a reference for the atmospheric pressure. \par
\noindent \textbf{Procedure:}  
\begin{enumerate}
    \item With the wind tunnel off (motor frequency of 0), measure the height of the liquid in each tube. 
    \item Increase the motor frequency by 5 hertz and wait for the liquid in the manometers to stabilize.   
    \item Record the new heights of the liquid in the manometers.  
    \item Repeat steps 2 and 3 until a motor frequency of 40 hertz is met.  
    \item Use the Matlab Script, {\it Lab2Analysis.m}, to plot the relationship between $q_T$ and $\Delta P$, find $K$, and plot the air velocity over the motor frequency. 
\end{enumerate}